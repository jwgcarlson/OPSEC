\documentclass{article}
\usepackage{amsmath}
\usepackage{amssymb}
\usepackage{bm}

\textwidth 6in
\oddsidemargin 0.25in
\parindent 12pt
\parskip 12pt

\renewcommand{\vec}[1]{\bm{#1}}
\newcommand{\mat}[1]{\textsf{#1}}
\newcommand{\nbar}{\bar{n}}
\DeclareMathOperator{\Cov}{Cov}
\DeclareMathOperator{\Tr}{Tr}

\begin{document}

{\Huge **Incomplete**}

\section*{General procedure}

Our input data is a collection of redshift-space galaxy positions
$(s_g,\theta_g,\phi_g)$ for $1 \le g \le N_\text{galaxies}$.  We collect this
data into a (redshift-space) galaxy number density
\begin{equation}
    n(\vec{s}) = \sum_g \delta_D(\vec{s} - \vec{s}_g),
\end{equation}
where $\delta_D$ is the Dirac delta function.  According to the standard
assumption, these galaxy positions are determined by a Poisson point process
with intensity function $\nbar (1 + \delta_g)$, where the galaxy density
contrast $\delta_g$ is linearly related to the underlying matter density
contrast $\delta$ on large scales.  Since we are only interested in the
large-scale behavior of $n(\vec{s})$, we project it onto a subspace spanned
by a suitable set of smooth functions over the survey volume $V$.

We introduce the standard inner product for functions over $\mathbb{R}^3$,
\begin{equation}
    (f, g) \equiv \int d^3x~ f(\vec{x}) g(\vec{x})
\end{equation}
We choose a set of basis functions $\{\phi_a(\vec{s})\}$ that span a reasonable
space of smooth functions over $V$, with $a \le 1 \le N_\text{basis}$, and
define basis pixels
\begin{equation}
    x_a \equiv (n, \phi_a)
        = \int d^3s~ n(\vec{s}) \phi_a(\vec{s})
        = \sum_g \phi_a(\vec{s}_g).
\end{equation}

Following the standard assumptions, we have
\begin{align}
    \langle n(\vec{s}) \rangle &= \nbar(\vec{s}), \\
    \langle n(\vec{s}_1) n(\vec{s}_2) \rangle &= \nbar(\vec{s}_1) \nbar(\vec{s}_2) [1 + \xi_g(\vec{s}_1,\vec{s}_2)] + \nbar(\vec{s}_1) \delta_D(\vec{s}_2 - \vec{s}_1).
\end{align}
where $\xi_g(\vec{s}_1,\vec{s}_2)$ is the (wide-angle, redshift-space) galaxy
correlation function.  The covariance of the pixel values $x_a$ is therefore
\begin{align}
    \Cov[x_a,x_b]
        &= \int d^3s_1~ d^3s_2~ \phi_a(\vec{s}_1) \phi_b(\vec{s}_2) \nbar(\vec{s}_1) \nbar(\vec{s}_2) \xi_g(\vec{s}_1,\vec{s}_2) + \int d^3s~ \phi_a(\vec{s}) \phi_b(\vec{s}) \nbar(\vec{s}) \\
        &\equiv S_{ab} + N_{ab}.
\end{align}
The first contribution $S_{ab}$ carries the cosmologically interesting signal,
while the second contribution $N_{ab}$ is simply Poisson noise.

While our original set of basis functions $\{\phi_a(\vec{s})\}$ may be chosen
arbitrarily, we are primarily interested in integral combinations of the galaxy
density that carry the most cosmological signal.  To that end, we first carry
out a Karhunen-Lo\`{e}ve transform to obtain a more suitable basis.  Let
$\{\lambda_i\}$ and $\{\vec{b}_i\}$ be the complete set of real eigenvalues and
orthonormal eigenvectors of the generalized eigenvalue problem $\mat{S} \vec{b}
= \lambda \mat{N} \vec{b}$, arranged in order of decreasing eigenvalue.  (A
complete set of such eigenvalues/eigenvectors is guaranteed to exist since $S$
is symmetric and $N$ is a positive definite matrix.  By ``orthonormal'' we mean
with respect to the noise matrix, i.e. $\vec{b}_i^T N \vec{b}_j =
\delta_{ij}$.)  Define a new basis of functions $\{\psi_i(\vec{s})\}$ by
\begin{equation}
    \psi_i(\vec{s}) = \sum_{a=1}^{N_\text{basis}} b_{i,a} \phi_a(\vec{s}),
\end{equation}
Then the new pixels defined by
\begin{equation}
    y_i \equiv (n, \psi_i) = \int d^3s~ n(\vec{s}) \psi_i(\vec{s}) = \sum_g \psi_i(\vec{s}_g)
\end{equation}
have covariance
\begin{equation}
    \Cov[y_i,y_j] = \delta_{ij} (1 + \lambda_i),
\end{equation}
so that the eigenvalue $\lambda_i$ measures the signal-to-noise ratio.

We refer to the functions $\psi_i$ as ``mode functions'' or ``KL mode
functions,'' and call the set of these mode functions the ``KL basis.'' Since
all we have done so far is perform a change of basis, we have lost no
information about the cosmological signal.  However, since the mode functions
are arranged in order of decreasing signal-to-noise, most of the interesting
signal is confined to the first several modes (where ``several'' may of course
be a large number, but hopefully is small compared to $N_\text{basis}$).  We
achieve a substantial data compression by simply discarding those modes with
lowest signal-to-noise ratio.  That is, we restrict the index $i$ to run only
from 1 to some $N_\text{modes}$.  We let $\mat{B}$ be the $N_\text{modes}
\times N_\text{basis}$ matrix with the eigenvectors $\vec{b}_i$ as rows, i.e.~
$B_{ia} = b_{i,a}$.  This matrix serves as a projection operator from the
original space of smooth functions to the KL subspace: 
$y_i = \sum_a B_{ia} x_a$ and $\Cov[y_i,y_j] = \sum_{a,b} B_{ia} B_{jb} \Cov[x_a,x_b]$.


\section*{Count-in-cell basis}

From the definitions of the pixels $x_a$ and the matrices $S_{ab}$ and
$N_{ab}$, we see that the basis functions $\phi_a(\vec{s})$ are always
multiplied by a factor of the selection function $\nbar(\vec{s})$, and
therefore the values of the basis functions outside the survey volume are
irrelevant.  In order to get the most out of our basis, without including
members, we should choose our functions $\phi_a$ to vanish outside the survey
volume.  Since modern surveys tend to have rather complicated angular
boundaries, the simplest way to enforce this constraint is to partition our
survey into a collection of discrete cells $V_a$, and choose our basis functions
to be indicator functions over these cells, i.e.
\begin{equation}
    \phi_a(\vec{s}) =
    \begin{cases}
        \bar{\phi}_a, & \vec{s} \in V_a, \\
        0, & \text{otherwise}.
    \end{cases}
\end{equation}
The constant value $\bar{\phi}_a$ is of course arbitrary, but we can achieve a
significant simplification if we choose $\bar{\phi}_a = \bar{N}_a^{-1/2}$ where
\begin{equation}
    \bar{N}_a = \int_{V_a} d^3s~ \nbar(\vec{s}),
\end{equation}
which fixes the noise matrix to be the identity:
\begin{equation}
    N_{ab} = \int d^3s~ \phi_a(\vec{s}) \phi_b(\vec{s}) \nbar(\vec{s}) = \delta_{ab}.
\end{equation}
Now the generalized eigenvalue problem $\mat{S} \vec{b} = \lambda \mat{N} \vec{b}$
reduces to the regular one, $\mat{S} \vec{b} = \lambda \vec{b}$.

\end{document}
