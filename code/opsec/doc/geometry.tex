\documentclass{article}
\usepackage{amsmath}
\usepackage{amssymb}
\usepackage{bm}

\textwidth 6in
\oddsidemargin 0.25in
\parindent 12pt
\parskip 12pt

\renewcommand{\vec}[1]{\bm{#1}}
\newcommand{\mat}[1]{\textsf{#1}}
\newcommand{\nbar}{\bar{n}}
\DeclareMathOperator{\Cov}{Cov}
\DeclareMathOperator{\Tr}{Tr}

\begin{document}

The 2-point correlation function depends on the geometry of the triangle formed
by the origin and the two sample points.  Parameterize this triangle by the
distance (from the origin) to the first point ($a$), the distance to the second
point ($b$), and the separation between the points ($s$).  Let the opening angle
at the origin be $2\theta$, and the other interior angles be $\alpha$ and $\beta$.

Under the plane-parallel approximation, we take the line-of-sight direction to
be the angle bisector between the two points, i.e. the line that splits the
opening angle $\theta$.  The following manipulations take us from the fiducial
geometrical description $(s,a,b)$ to the plane-parallel values $(s,\mu)$, where
$\mu = \cos\phi$ with $\phi$ the angle between the line-of-sight direction and
the pair separation.

\begin{gather*}
    \cos2\theta = \frac{a^2 + b^2 - s^2}{2ab} \\
    \cos A = \frac{s^2 + b^2 - a^2}{2sb} \\
    \cos B = \frac{s^2 + a^2 - b^2}{2sa} \\
    \phi = A + \theta \\
    \mu = \cos\phi = \cos(A + \theta) = \cos A \cos\theta - \sin A \sin\theta
\end{gather*}


\end{document}
