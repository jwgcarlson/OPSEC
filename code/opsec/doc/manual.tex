\documentclass{article}
\usepackage{amsmath}
\usepackage{bm}
\usepackage{color,graphicx}
\usepackage{hyperref}
\usepackage{xspace}

\textwidth 6in
\oddsidemargin 0.25in
\parskip 12pt
\parindent 12pt

\newcommand{\opsec}{\protect{\textrm{OPSEC}}\xspace}
\newcommand{\version}{\protect{\textrm{v0.1}}\xspace}
\newcommand{\homepage}{\protect{\texttt{HOMEPAGE}}\xspace}
\newcommand{\basis}{\protect{\texttt{basis}}\xspace}
\newcommand{\klt}{\protect{\texttt{klt}}\xspace}
\newcommand{\comma}{\protect{\texttt{comma}}\xspace}
\renewcommand{\dot}{\protect{\texttt{dot}}\xspace}
\newcommand{\estimate}{\protect{\texttt{estimate}}\xspace}
\newcommand{\eg}{\textit{e.g.}}
\newcommand{\ie}{\textit{i.e.}}
\newcommand{\etal}{\textit{et.~al.}}
\renewcommand{\vec}[1]{\bm{#1}}
\newcommand{\mat}[1]{\mathbf{#1}}
\renewcommand{\r}{\vec{r}}
\renewcommand{\phi}{\varphi}
\newcommand{\RMin}{\mathtt{RMin}}
\newcommand{\RMax}{\mathtt{RMax}}
\newcommand{\MuMin}{\mathtt{MuMin}}
\newcommand{\MuMax}{\mathtt{MuMax}}
\newcommand{\PhiMin}{\mathtt{PhiMin}}
\newcommand{\PhiMax}{\mathtt{PhiMax}}
\newcommand{\Nr}{\mathtt{Nr}}
\newcommand{\Nmu}{\mathtt{Nmu}}
\newcommand{\Nphi}{\mathtt{Nphi}}
\newcommand{\Shat}{\widehat{S}}
\newcommand{\Nhat}{\widehat{N}}
\DeclareMathOperator{\Tr}{Tr}
\DeclareMathOperator{\Cov}{Cov}
\DeclareMathOperator{\Var}{Var}

\title{\opsec \version Manual}
\date{23 August 2011}

\begin{document}
\maketitle

This manual describes the usage of the \opsec (Optimal Power Spectrum
Estimation Code).  The complete source code, as well as the latest version of
this manual, are available at \homepage.

Section \ref{sec:formalism} provides theoretical background to understand
precisely what it is that \opsec computes.  Section \ref{sec:implementation}
describes how these theoretical ideas are implemented.  Readers who just want
to learn how to use the code can skip these sections and move straight to
Section \ref{sec:build}, which gives instructions on how to compile the \opsec
code.  Once compiled, Section \ref{sec:quickstart} shows how to perform a full
estimation run for a simplified problem.  More detailed information about the
various programs and utilities that comprise \opsec is provided in Section
\ref{sec:codedetails}.

\tableofcontents

\section{Formalism for Optimal Quadratic Estimators}
\label{sec:formalism}

\opsec was designed primarily to measure the power spectrum of galaxies,
specifically for the SDSS-II and SDSS-III catalogs of luminous red galaxies
(LRGs).  However, the formalism of optimal quadratic estimation that it follows
is much more general.  With the proper choice of inputs, \opsec can be used to
estimate any cosmological parameters from the clustering signal of any set of
astrophysical objects.  In this section we describe this broader formalism from
a high-level, theoretical point of view.  In later sections we specialize to
the case of measuring $P(k)$ from the large-scale clustering of galaxies.

To make use of \opsec, only three inputs must be given:
\begin{enumerate}
\item The observed number density field $n(\r)$ for a particular tracer
    (equivalently, a list of positions $\{\r_g\}$).
\item The expected density $\bar{n}(\r)$ of this tracer in the absence of
    clustering (usually estimated from the observed positions, taking into
    account the selection and masks used in the survey).
\item A model for how the observed 2-point correlation function depends on a
    set of cosmological parameters $p_n$ ($1 \le n \le N_p$), along with a set
    of prior estimates $p_n^{(0)}$ for these parameters (usually from previous
    measurements).
\end{enumerate}
These inputs are used to construct optimal estimates $\hat{p}_n$ for the
parameters $p_n$, as follows.  (The precise sense in which these estimates are
``optimal'' is described in Section \ref{sec:optimal}.)

First the observed density field $n(\r)$ is collapsed into a set of real-valued
coefficients $y_i$, by choosing a suitable set of mode functions
$\psi_i(\r)$, and defining
\begin{equation}
    y_i = \int d^3r~ \psi_i(\r) [n(\r) - \bar{n}(\r)].
\end{equation}
(Here and in the following, 3-dimensional integrals are to extend only over the
volume of the survey.)  We refer to these coefficients $y_i$ as ``pixel
values.''  The optimal choice of the mode functions $\psi_i$ is itself a
complicated issue, which we address in Section \ref{sec:klmodes}.  For now,
they may be viewed as an arbritrary set of $N_m$ linearly independent functions
defined over the survey volume.  Regardless of the choice of mode functions,
the pixel values $y_i$ have zero mean: $\langle y_i \rangle = 0$.  (This is
precisely true only if the mean number density is known exactly; if $\bar{n}$
is estimated from data, this condition will hold only approximately.)

Our tracers are presumed to be discrete objects (\eg~galaxies), and therefore
their observed clustering pattern is subject to Poisson shot noise:
\begin{equation}
    \langle n(\r_1) n(\r_2) \rangle =
        \bar{n}(\r_1) \bar{n}(\r_2) [1 + \xi(\r_1,\r_2)]
      + \bar{n}(\r_1) \delta_D(\r_2-\r_1)
\end{equation}
where $\delta_D$ is a Dirac delta function and $\xi(\r_1,\r_2)$ is the 2-point
correlation function.  The covariance of the pixel values can therefore be
written as the sum of a signal component and a noise component,
\begin{equation}
\label{eq:yiyj}
    \langle y_i y_j \rangle = C_{ij} = S_{ij} + N_{ij},
\end{equation}
where
\begin{align}
\label{eq:signal}
    S_{ij} &= \int d^3r_1 \int d^3r_2~ \psi_i(\r_1) \psi_j(\r_2)~ \bar{n}(\r_1) \bar{n}(\r_2)~ \xi(\r_1,\r_2), \\
\label{eq:noise}
    N_{ij} &= \int d^3r~ \psi_i(\r) \psi_j(\r)~ \bar{n}(\r).
\end{align}

We have defined the pixel values to have zero mean, since we are interested
only in cosmological parameters that depend on the \emph{clustering} of our
tracers.  Thus the lowest order statistics that can be meaningfully related
to our parameters are quadratic combinations
\begin{equation}
    \hat{p} = A_{ij} y_i y_j + b,
\end{equation}
where summation over repeated indices is implied.  We will frequently view
$A_{ij}$ as the components of a matrix, and $y_i$ as components of a vector,
and write $\hat{p} = \mat{y}^T \mat{A} \mat{y} + b$.

Under ideal circumstances (see Appendix \ref{sec:optimal} for details and
caveats), it can be shown that the minimum variance, unbiased quadratic
estimators for the parameters $p_n$ are
\begin{equation}
\label{eq:pm}
    \hat{p}_m = \sum_n (F^{-1})_{mn} \left(q_n - f_n \right)
\end{equation}
where
\begin{align}
\label{eq:qn}
    q_n &= \frac{1}{2} \mat{y}^T \mat{C}^{-1} \mat{C}_{,n} \mat{C}^{-1} \mat{y}, \\
\label{eq:fn}
    f_n &= \frac{1}{2} \Tr\left[ \mat{C}^{-1} \mat{C}_{,n} \mat{C}^{-1} \mat{N}\right], \\
\label{eq:Fmn}
    F_{mn} &= \frac{1}{2} \Tr\left[ \mat{C}^{-1} \mat{C}_{,m} \mat{C}^{-1} \mat{C}_{,n}\right],
\end{align}
and $\mat{C}_{,n}$ denotes the partial derivative
$\partial \mat{C}/\partial p_n$.  The matrices $\mat{C}$ and $\mat{C}_{,n}$
here are to be evaluated at $p_n = p_n^{(0)}$, \ie~with the prior parameter
estimates.  $F_{mn}$ is the Fisher matrix which (as with all minimum variance
estimators) determines the errors on $\hat{p}_m$,
\begin{equation}
    \Cov[\hat{p}_m,\hat{p}_n] = (F^{-1})_{mn}.
\end{equation}


\subsection{Optimal mode functions}
\label{sec:klmodes}

The choice of the mode functions $\psi_i(\r)$ affects the accuracy of the
estimates $\hat{p}_n$.  The optimal choice depends on the parameters that are
to be estimated, the geometry of the survey, and the number density of the
desired tracers.  Following Tegmark \etal, we choose the mode functions to be
signal-to-noise eigenmodes.  That is, we choose mode functions so that the
signal and noise matrices (see Eqs. (\ref{eq:signal}-\ref{eq:noise})) are
simultaneously diagonal: $S_{ij} = s_i \delta_{ij}$ and $N_{ij} = n_i
\delta_{ij}$ (no sum on $i$).  Of the infinite number of possible mode
functions, we choose a finite number $N_m$ having the largest signal-to-noise
eigenvalue $\lambda_i = s_i/n_i$.

More explicitly, these signal-to-noise eigenmodes are constructed through the
following procedure.  First, we choose a convenient basis of linearly
independent functions $\{\phi_a(\r)\}$ defined over the survey geometry, and
compute the signal and noise matrices in this basis,
\begin{align}
    \Shat_{ab} &= \int d^3r \int d^3r'~ \phi_a(\r) \phi_b(\r')~ \bar{n}(\r) \bar{n}(\r')~ \xi(\r,\r'), \\
    \Nhat_{ab} &= \int d^3r~ \phi_a(\r) \phi_b(\r)~ \bar{n}(\r).
\end{align}
Next we solve the generalized eigenvalue problem
\begin{equation}
    \mat{\Shat} \mat{b} = \lambda \mat{\Nhat} \mat{b},
\end{equation}
obtaining a complete set of eigenvectors $\{\mat{b}_i\}$ and associated
eigenvalues $\{\lambda_i\}$.  (This is possible since $\Shat_{ab}$ and
$\Nhat_{ab}$ are real symmetric matrices, and $\Nhat_{ab}$ is
positive-definite.)  Then the functions
\begin{equation}
    \psi_i(\r) = \sum_a b_{ia} \phi_a(\r)
\end{equation}
are signal-to-noise eigenmodes.  For convenience, we normalize the $\psi_i$ so
that they are noise-orthonormal, \ie~so that the noise matrix is the identity:
$N_{ij} = \delta_{ij}$.  Then the covariance of pixel values in Eq.
(\ref{eq:yiyj}) becomes simply
\begin{equation}
    \langle y_i y_j \rangle = C_{ij} = (1 + \lambda_i) \delta_{ij} \quad \text{(no sum on $i$)}.
\end{equation}

At this point, there are as many mode functions $\psi_i$ as there are initial
basis functions $\phi_a$.  However, many of these functions contribute very
little information to the parameter estimates $\hat{p}_n$, in that the
expectation of the quadratic combination $y_i^2$ is dominated by noise rather
than the desired signal.  Thus, by keeping only modes with the largest
signal-to-noise eigenvalues $\lambda_i$, we can reduce the size of the
computational problem without sacrificing too much information.  In practice we
first fix the desired number of modes $N_m$ (the choice of which is informed by
the desired error properties on our parameter estimates), then sort the
eigenmodes by their signal-to-noise eigenvalue $\lambda_i$, and keep only the
top $N_m$ mode functions $\psi_i$.

This procedure is referred to by Tegmark \etal~as \emph{Karhunen-Lo\`{e}ve
compression}.  The mode functions $\psi_i$ are then referred to as
Karhunen-Lo\`{e}ve mode functions, or simply KL modes.


\subsection{Examples}

\subsection{Derivation of optimal quadratic estimator}
\label{sec:derivation}

Here we show how the quadratic estimator of Eq. (\ref{eq:pm}) is derived.  As
our starting point, we note that the true covariance matrix of pixel values
$C_{ij} = \langle y_i y_j \rangle$ is unknown.  However, if our model is
accurate and our prior estimates $p_n^{(0)}$ for the unknown parameter values
$p_n$ are close, then we can expand the true covariance matrix around our prior
estimate for it,
\begin{equation}
    C_{ij} = C_{ij}^{(0)} + \sum_{n=1}^{N_p} C_{ij,n}^{(0)} (p_n - p_n^{(0)}) + \dots,
\end{equation}
where $\mat{C}^{(0)}$ and $\mat{C}_{,n}^{(0)}$ are the covariance matrix and
its derivatives calculated using the prior estimates $p_n^{(0)}$.  We have
truncated the Taylor expansion at first order, which is a good approximation if
the prior estimates $p_n^{(0)}$ are close to the true values $p_n$.  (In fact,
if $\mat{C}$ depends linearly on the $p_n$, this expansion is exact.)

We want to construct estimators
\begin{equation}
    \hat{p}_m = A_{mij} y_i y_j + b_m
\end{equation}
that are unbiased, $\langle \hat{p}_m \rangle = p_m$, and for which the
variance is minimized.  (Throughout this section, summation is implied for
repeated indices $i,j,k,l$, while sums over $m$ or $n$ are always written out
explicitly.) Given the above Taylor expansion for $C_{ij}$, we have
\begin{equation}
    \langle \hat{p}_m \rangle 
        = A_{mij} \sum_n C_{ij,n}^{(0)} p_n + A_{mij} \left[C_{ij}^{(0)} - \sum_n C_{ij,n}^{(0)} p_n^{(0)}\right] + b_m.
\end{equation}
So for $\hat{p}_m$ to be unbiased, the coefficients $A_{mij}$ must satisfy
\begin{equation}
    \label{eq:unbias}
    A_{mij} C_{ij,n}^{(0)} = \Tr\left[\mat{A}_m \mat{C}_{,n}^{(0)}\right] = \delta_{mn},
\end{equation}
while $b_m$ is fixed to be
\begin{equation}
    b_m = -A_{mij} \left[C_{ij}^{(0)} - \sum_n C_{ij,n}^{(0)} p_n^{(0)}\right] = -\Tr\left[\mat{A}_m \left(\mat{C}^{(0)} - \sum_n \mat{C}_{,n}^{(0)} p_n^{(0)}\right)\right].
\end{equation}

We now seek to minimize the variance
\begin{equation}
    \label{eq:varpm}
    \Var[\hat{p}_m] = A_{mij} A_{mkl} R_{ijkl}
\end{equation}
where we have defined
\begin{equation}
    R_{ijkl} = \langle y_i y_j y_k y_l \rangle - \langle y_i y_j \rangle \langle y_k y_l \rangle.
\end{equation}
If we assume the pixel values $y_i$ obey Gaussian statistics (see next section
for more on this assumption), then
\begin{equation}
    R_{ijkl} = C_{ik} C_{jl} + C_{il} C_{jk}.
\end{equation}
To minimize (\ref{eq:varpm}) subject to the constraint (\ref{eq:unbias}), we
introduce Lagrange multipliers $\lambda_n$ and define
\begin{equation}
    S = A_{mij} A_{mkl} (C_{ik} C_{jl} + C_{il} C_{jk}) + \sum_n \lambda_n \left(A_{mij} C_{ij,n}^{(0)} - \delta_{mn}\right).
\end{equation}
Locating the extremum of this quantity yield the conditions
\begin{align}
    \frac{\partial S}{\partial A_{mij}} &= A_{mkl} R_{ijkl} + \sum_n \lambda_n C_{ij,n}^{(0)} = 0, \\
    \frac{\partial S}{\partial \lambda_n} &= A_{mij} C_{ij,n}^{(0)} - \delta_{mn} = 0,
\end{align}
which admits the solution
\begin{equation}
    \mat{A}_m = \sum_n (F^{-1})_{mn} \frac{1}{2} \mat{C}^{-1} \mat{C}_{,n}^{(0)} \mat{C}^{-1},
\end{equation}
where the Fisher matrix $F_{mn}$ is given by Eq.~(\ref{eq:Fmn}).

(depends on true Cij, which is unknown...)

\subsection{Optimality of quadratic estimator}
\label{sec:optimal}

- Discuss conditions under which estimator is optimal: priors and model are
close to truth, and (ideally) signal matrix is linear in parameters. \\
- Show derivation using variance functional and Lagrange multipliers. \\
- Discuss choice of mixing matrix, explaining how $M$ can help reduce
correlations while only slightly increasing variance.


\section{Implementation Notes}
\label{sec:implementation}


\subsection{Coordinate systems}

\opsec supports two coordinate systems: spherical and Cartesian.  In Cartesian
mode, positions $\r$ are represented by a triple $(x,y,z)$.  In spherical mode,
positions are represented by a triple $(r,\mu,\phi)$, where $\mu = \cos\theta$
is the cosine of the polar angle.  The translation to or from Cartesian
coordinates is given by
\begin{gather}
    x = r \sqrt{1-\mu^2} \cos\phi, \qquad y = r \sqrt{1-\mu^2} \sin\phi, \qquad z = r \mu, \\
    r = \sqrt{x^2+y^2+z^2}, \qquad \mu = z/r, \qquad \phi = \tan^{-1} y/x.
\end{gather}


\subsection{Cell basis}
\label{sec:cells}

As explained in Section \ref{sec:klmodes}, in order to compute the KL mode
functions $\psi_i(\r)$ we must first choose a basis of functions $\phi_a(\r)$
over the survey geometry.  For \opsec, we partition the survey geometry into a
mesh of regularly-shaped cells, and choose our basis functions to be constant
over each cell (\ie~$\phi_a(\r)$ is proportional to the indicator function over
cell $a$).  A detailed explanation is given below for the case of spherical
coordinates; the Cartesian case is analagous.

First, a regular bounding region is defined, such that the entire survey fits
within the region
\begin{equation}
    \{(r,\mu,\phi) : \RMin \le r \le \RMax, \MuMin \le \mu \le \MuMax, \PhiMin \le \phi \le \PhiMax\}. \\
%[\RMin,\RMax]\times[\MuMin,\MuMax]\times[\PhiMin,\PhiMax]
\end{equation}
This region is then subdivided into cells, by partitioning each coordinate
interval into equal subintervals:
\begin{itemize}
\item $[\RMin,\RMax]$ is divided into $\Nr$ equal subintervals $[r_d,r_{d+1}]$ for $0 \le d < \Nr$.
\item $[\MuMin,\MuMax]$ is divided into $\Nmu$ equal subintervals $[\mu_e,\mu_{e+1}]$ for $0 \le e < \Nmu$.
\item $[\PhiMin,\PhiMax]$ is divided into $\Nphi$ equal subintervals $[\phi_f,\phi_{f+1}]$ for $0 \le f < \Nphi$.
\end{itemize}
(See Figure \ref{fig:cells}.)  This subdivision of the bounding region gives
a grid of $\Nr*\Nmu*\Nphi$ cells, each labeled by a triple $(d,e,f)$:
\begin{equation}
    C_{d,e,f} = [r_d,r_{d+1}]\times[\mu_e,\mu_{e+1}]\times[\phi_f,\phi_{f+1}].
\end{equation}
Note that, although each cell has the same thickness and the same angular extents,
\begin{equation}
    \Delta r = \frac{\RMax-\RMin}{\Nr}, \quad
    \Delta\mu = \frac{\MuMax-\MuMin}{\Nmu}, \quad
    \Delta\phi = \frac{\PhiMax-\PhiMin}{\Nphi},
\end{equation}
their volumes will differ due to the varying radial positions:
\begin{equation}
    V_{d,e,f} = \text{Vol } C_{d,e,f}
    = \frac{1}{3} (r_{d+1}^3-r_d^3) \Delta\mu \Delta\phi.
\end{equation}
(This is true only for the spherical case; for Cartesian coordinates all cells
are identical.)

Most surveys are not regularly shaped, and consequently many of the cells in
this grid will be wholly outside of the survey region.  Rather than carrying
these empty cells through our calculations, we discard them right away.
Explicitly, we first order the cells within the grid by a grid index
\begin{equation}
    G = (d*\Nmu + e)*\Nphi + f, \quad 0 \le G < \Nr*\Nmu*\Nphi.
\end{equation}
We then take all $N_\text{cells}$ non-empty cells and label them by a
consecutive index $a$, $0 \le a < N_\text{cells}$, according to the ordering
imposed by $G$.  Thus each non-empty cell can be labelled either by the cell
index $a$, by the grid index $G$, or by the grid coordinates $(d,e,f)$.

For each non-empty cell $C_a$ we compute the number of galaxies expected to lie
within the cell according to the given selection function,
\begin{equation}
    \bar{N}_a = \int_{C_a} d^3r~ \bar{n}(\r).
\end{equation}
We also compute the effective volume of the cell, $V_{a,\text{eff}}$, which is
defined as the volume of the cell for which $\bar{n}(\r)$ is non-vanishing.
(This isn't used for anything yet.)

For our basis functions, finally, we define
\begin{equation}
    \phi_a(\r) = \bar{N}_a^{-1/2} \chi_a(\r) = \begin{cases} \bar{N}_a^{-1/2} & \text{if } \r \in C_a, \\ 0 & \text{otherwise}. \end{cases}
\end{equation}
The normalization factor $\bar{N}_a^{-1/2}$ is chosen so that the noise matrix
is $\Nhat_{ab} = \delta_{ab}$.  The signal matrix in this basis is then given
by
\begin{align}
    \Shat_{ab} &= \bar{N}_a^{-1/2} \bar{N}_b^{-1/2} \int_{C_a} d^3r \int_{C_b} d^3r'~ \bar{n}(\r) \bar{n}(\r')~ \xi(\r,\r') \\
    &\approx \bar{N}_a^{1/2} \bar{N}_b^{1/2} \int_{C_a} \frac{d^3r}{V_a} \int_{C_b} \frac{d^3r'~}{V_b} \xi(\r,\r').
\end{align}
The latter approximation is used in \opsec as it leads to a significant
optimization (see Appendix \ref{sec:symmetries}); it is a good approximation as
long as the cells are small compared to the scale of density fluctuations we
are trying to probe.

\subsection{Surveys}
\label{sec:survey}

In the \opsec framework, a \emph{survey} is characterized by two pieces of
information:
\begin{enumerate}
\item a \emph{selection function} $\bar{n}(\r)$, which gives the expected
    number density of galaxies at position $\r$ in the absence of clustering.
    This function implicitly defines the survey geometry ($\bar{n}(\r) = 0$
    everywhere outside the survey).
\item a list of \emph{galaxies} (or other tracers).  A galaxy is defined by
    its position (either $x,y,z$ in Cartesian mode, or $r,\mu,\phi$ in
    spherical mode) and by a weight $w$ (used as a fudge factor by survey
    teams to deal with issues such as fiber collision).
\end{enumerate}
A survey in \opsec is implemented as a subclass of the abstract base class
\texttt{Survey}, declared in \texttt{opsec/src/Survey.h}.  See Section
\ref{sec:implsurvey} for a more detailed description.


\subsection{Models}

In Section \ref{sec:formalism}, we said that the final input that \opsec needs
is a model for how the observed 2-point correlation function depends on the
parameters $p_n$, as well as a set of prior estimates $p_n^{(0)}$.  More
precisely, \opsec requires a prior estimate for the 2-point correlation
function $\xi(\r_1,\r_2)$, as well as estimates for the functions
$\xi_{,n}(\r_1,\r_2)$, where as usual commas denote derivatives with respect
parameters: $\xi_{,n} = \partial \xi/\partial p_n$.  These prior estimates are
used in the construction of the matrices $\mat{C}$ and $\mat{C}_{,n}$ in
Eqs. (\ref{eq:qn}), (\ref{eq:fn}), and (\ref{eq:Fmn}).  Explicitly, the signal
components of these matrices are given by
\begin{align}
    S_{ij} &= \int d^3r_1 \int d^3r_2~ \psi_i(\r_1) \psi_j(\r_2)~ \bar{n}(\r_1) \bar{n}(\r_2)~ \xi(\r_1,\r_2), \\
    S_{ij,n} &= \int d^3r_1 \int d^3r_2~ \psi_i(\r_1) \psi_j(\r_2)~ \bar{n}(\r_1) \bar{n}(\r_2)~ \xi_{,n}(\r_1,\r_2).
\end{align}
The prior estimate for $\xi(\r_1,\r_2)$ is also used in the construction of the
KL mode functions $\psi_i(\r)$.


\subsection{Geometry of 2-point clustering}

For a statistically isotropic field (which describes nearly every possible
field in cosmology, including especially the redshift space galaxy number
density), the 2-point correlation function $\xi(\r_1,\r_2)$ depends only on
the geometry of the triangle formed by the observer $\vec{O}$ and the two
positions $\r_1$, $\r_2$.

Of the many possible ways of parameterizing this geometry, in \opsec we use the
three distances $r = |\r_2-\r_1|$, $a = |\r_1-\vec{O}|$, and $b=|\r_2-\vec{O}|$
(see Figure \ref{fig:geometry}).  Thus, the correlation function and its
derivatives are actually implemented as functions of these three scalars:
$\xi(\r_1,\r_2) = \xi(r,a,b)$ and $\xi_{,n}(\r_1,\r_2) = \xi_{,n}(r,a,b)$.

\begin{figure}
\documentclass{article}
\usepackage{amsmath}
\usepackage{amssymb}
\usepackage{bm}

\textwidth 6in
\oddsidemargin 0.25in
\parindent 12pt
\parskip 12pt

\renewcommand{\vec}[1]{\bm{#1}}
\newcommand{\mat}[1]{\textsf{#1}}
\newcommand{\nbar}{\bar{n}}
\DeclareMathOperator{\Cov}{Cov}
\DeclareMathOperator{\Tr}{Tr}

\begin{document}

The 2-point correlation function depends on the geometry of the triangle formed
by the origin and the two sample points.  Parameterize this triangle by the
distance (from the origin) to the first point ($a$), the distance to the second
point ($b$), and the separation between the points ($s$).  Let the opening angle
at the origin be $2\theta$, and the other interior angles be $\alpha$ and $\beta$.

Under the plane-parallel approximation, we take the line-of-sight direction to
be the angle bisector between the two points, i.e. the line that splits the
opening angle $\theta$.  The following manipulations take us from the fiducial
geometrical description $(s,a,b)$ to the plane-parallel values $(s,\mu)$, where
$\mu = \cos\phi$ with $\phi$ the angle between the line-of-sight direction and
the pair separation.

\begin{gather*}
    \cos2\theta = \frac{a^2 + b^2 - s^2}{2ab} \\
    \cos A = \frac{s^2 + b^2 - a^2}{2sb} \\
    \cos B = \frac{s^2 + a^2 - b^2}{2sa} \\
    \phi = A + \theta \\
    \mu = \cos\phi = \cos(A + \theta) = \cos A \cos\theta - \sin A \sin\theta
\end{gather*}


\end{document}

\end{figure}


\section{Code Usage}
\label{sec:codeusage}

\subsection{Programs}

The core functionality of \opsec is broken into several independent programs.
Each program performs a single logical operation, with well-defined input and
output.  Structuring the code in this way has a number of advantages:
\begin{itemize}
\item it's easier to develop, since each program performs a single task or a
    series of closely related tasks;
\item it's easier to debug, since each program can be tested individually;
\item it allows parts of the code to be re-run without having to completely
    restart the analysis (\eg~the same KL mode functions can be used for
    different model parameterizations).
\end{itemize}

In the following subsections we describe each program in more detail.

\subsubsection{\basis}

The \basis program is responsible for producing properly normalized basis
functions $\phi_a(\r)$.  For input it takes a specification of the survey,
including its geometry and selection function, and for output it produces a
list of non-empty cells (see Section \ref{sec:cells}).  A cell is described by
its cell index $a$, its coordinate bounds (min and max $r$, $\mu$, and $\phi$
values in the spherical case), and the number of galaxies $\bar{N}_a$ expected
to lie within the cell.

\subsubsection{\klt}

The \klt program computes the signal-to-noise eigenvalues $\lambda_i$, and
the corresponding eigenmodes $\psi_i(\r)$ in terms of the basis functions
$\phi_a(\r)$.

\subsubsection{\comma}

The \comma program computes the derivatives $C_{ij,n}$ of the covariance matrix
with respect to parameters.

\subsubsection{\dot}

The \dot program computes the pixel values $y_i$ using actual galaxy positions.

\subsubsection{\estimate}

The \estimate program performs the necessary linear algebra operations to
obtain the minimum variance parameter estimates $\hat{p}_m$.

\subsection{Annotated binary format}

At various points in its operation \opsec needs to store large, dense numerical
arrays (vectors or matrices) on disk.  A binary format is necessary both to
minimize disk usage and to ensure fast reading and writing of the data.  \opsec
uses \emph{annotated binary} files, which generally carry a \texttt{.abn}
suffix.  A \texttt{.abn} file consists of a text header, followed by a special
``tag'' line describing the data, followed by the raw binary data itself.  The
text header can contain arbitrary text, which provides a convenient mechanism
to document the origins and contents of the data.  The tag line is of the form
\begin{equation}
    \texttt{\#!Begin data block: n = N, size = SZ, endian = E, format = FMT}
\end{equation}
where \texttt{N} indicates the number of data elements, \texttt{SZ} indicates
the number of bytes per data element, \texttt{E} is a single character
indicating whether the binary data is stored in little-endian (\texttt{L}) or
big-endian (\texttt{B}) form, and \texttt{FMT} is a string that describes the
intended type for the data elements.  The format string follows the conventions
of the Python \texttt{struct} module, described here:
\url{http://docs.python.org/library/struct.html#format-characters}.

\subsection{Configuration files}

\opsec is a fairly flexible system, and as such it needs to read in several
parameter values so that it knows what to compute.  It reads these values from
simple ASCII configuration files, usually given a \texttt{.cfg} suffix.  Each
configuration file is a simple text file consisting of ``name = value'' pairs,
where blank lines or lines beginning with the `\texttt{\#}' character are
ignored.

\subsection{Configuration options for \opsec programs}

[List all required/accepted configuration options for each program, with a
brief explanation of each.]

\subsection{Utility programs}


\section{Code Details}
\label{sec:codedetails}


\section{Building and Running \opsec}
\label{sec:build}


\begin{appendix}

\end{appendix}

\end{document}
